\section{Polygonkurver}
En polygonkurve er den kurve der fremkommer når et antal hjørnepunkter forbindes med rette linjer. Vi vil i denne del af rapporten vise hvordan længden af en polygonkurve kan beregnes, og bruge polygonkurver til at finde geodæter på de flader der tidligere viste sig at være problematiske at løse.
\subsection{Længden af en polygonkurve}
Placeres, og forbindes, \(n\) punkter, \(t_1...t_n\) på kurven \(\boldsymbol{\gamma}(t)=(x(t),y(t),z(t))\), der er defineret i intervallet \([a;b]\), fås en polygonkurve \(\boldsymbol{\gamma}_p\) med \(n\) hjørnepunkter. Hjørnepunktet \(p_i\) vil have koordinatsættet \(\boldsymbol{\gamma}(t_i)\). Længden af den originale kurve, \(\boldsymbol{\gamma}(t)\), for \(t ~ \in ~ [a;b]\) er som bekendt
\begin{equation}
\mathscr{L}(\boldsymbol{\gamma}(t)) = \int\limits_a^b ||\dot{\boldsymbol{\gamma}}(t)||
\label{curvelength}
\end{equation}
Mens længden af den fremkomne polygonkurve er 
\begin{equation}
\ell _n = \sum\limits_{i=2}^n \sqrt{(x(t_i)-x(t_{i-1}))^2+(y(t_i)-y(t_{i-1}))^2+(z(t_i)-z(t_{i-1}))^2}
\label{polycurvelength}
\end{equation} 
Eftersom polygonkurven approksimerer den originale kurve, vil længden af polygonkurven nærme sig længden af den originale kurve som antallet af hjørnepunkter stiger. Dvs.
\begin{equation}
\ell_n \rightarrow \mathscr{L}(\gamma(t)) \quad \mbox{for} \quad n \rightarrow \infty
\label{polygoestowardscurve}
\end{equation}
Ved hjælp af middelværdisætningen kan udtrykket for polygonkurvens længde omskrives således at x,y og z funktionernes afledte bruges til at finde længden af polygonkurven. \\
\\
Først multipliceres udtrykket for \(\ell_n\) med \(\frac{t_i-t_{i-1}}{t_i-t_{i-1}}\), hvorved opnås:

\begin{equation}
\ell _n = \sum\limits_{i=2}^n \sqrt{\left(\frac{x(t_i)-x(t_{i-1})}{t_i-t_{i-1}}\right)^2+\left(\frac{y(t_i)-y(t_{i-1})}{t_i-t_{i-1}}\right)^2+\left(\frac{z(t_i)-z(t_{i-1})}{t_i-t_{i-1}}\right)^2}\cdot(t_i-t_{i-1})
\end{equation}
Herefter anvendes middelværdisætningen \\

\paragraph{Middelværdisætningen:} 
Hvis en funktion \(f\) er kontinuert og differentiabel i et interval \([a;b]\) og differentiabel i intervallet \(]a;b[\), så findes der et punkt \(c\) i \(]a;b[\) således at
\begin{equation}
f\prime(c) = \frac{f(b)-f(a)}{b-a}
\end{equation}
Og der kan reduceres til
\begin{equation}
 \ell _n = \sum\limits_{i=2}^n \sqrt{ \dot{x}(\zeta_i)^2 +\dot{y}(\eta_i)^2 + \dot{z}(\xi_i)^2}(t_i-t_{i-1}), \quad (\zeta_i, \eta_i, \xi_i) ~ \in ~ ]t_{i-1};t_i[
 \label{ellndiff}
\end{equation}
Polygonkurvens længde kan også approksimeres ved at opstille midtsummen for \(\ell_n\).\\
Midtsummen for en funktion \(f\) defineret i et interval \([a;b]\) er givet ved
\begin{equation}
S=Q\left(f\left(a+\frac{Q}{2}\right) + f\left(a+\frac{3Q}{2}\right) + \cdot\cdot\cdot + f\left(b-\frac{Q}{2}\right)\right)
\label{midtsumref}
\end{equation}
Dette kan skrives på sædvanlig sum-form som
\begin{equation}
S=\sum\limits_{n=0}^{n=k} \left(f\left(a+\frac{(1+2n)Q}{2}\right)\right)
\label{midtsumsum}
\end{equation}
Her afhænger konstanten \(k\) af det valgte \(Q\), der kan ses som finheden, da \(Q\) bestemmer skridtstørrelsen der tages inden for det givne interval. Ifølge (\ref{midtsumref}) må følgene nødvendigvis gøre sig gældende i (\ref{midtsumsum})
\begin{equation}
\begin{gathered}
a+\frac{(1+2k)Q}{2} =b-\frac{Q}{2} \\
\Updownarrow \\
a+\frac{kQ}{2} = b-Q \\
\Updownarrow \\
k = \frac{b-a}{Q}-1
\end{gathered}
\end{equation}
Anvender vi (\ref{midtsumsum}) på udtrykket for kurvelængden (\ref{curvelength}) har vi
\begin{equation}
S_n = Q\sum\limits_{n=0}^{n=\frac{b-a}{Q}-1}\sqrt{\dot{x}^2\left(a+\frac{(1+2n)Q}{2}\right)+\dot{y}^2\left(a+\frac{(1+2n)Q}{2}\right)+\dot{z}^2\left(a+\frac{(1+2n)Q}{2}\right)}
\end{equation}
Dette ligner udtrykket for \(\ell_n\) i (\ref{ellndiff}), hvor der her bare er valgt et fast antal ækvidistante punkter mellem \(a\) og \(b\) som følge af \(Q\) der bestemmer afstanden mellem to på hinanden følgene punkter.
\\
Ønsker vi også at have ækvidistance punkter, i et interval \([a;b]\), i udtrykket for \(\ell_n\) (\ref{polycurvelength}) får vi
\begin{equation}
\ell _n = \sum\limits_{i=1}^{n=k(b-a)} \sqrt{\left(x\left(a+\frac{i}{k}\right)-x\left(a+\frac{i-1}{k}\right)\right)^2+\left(y\left(a+\frac{i}{k}\right)-y\left(a+\frac{i-1}{k}\right)\right)^2+\left(z\left(a+\frac{i}{k}\right)-z\left(a+\frac{i-1}{k}\right)\right)^2}
\end{equation}
Ønskes der samme finhed for \(\ell_n\) og \(S_n\) gælder følgene sammenhæng:
\begin{equation}
Q=\frac{1}{k}
\end{equation}
Da \(S_n\) er en approksimation af \(\ell_n\) er det naturligt at termerne vil nærme sig hinanden som finheden af deres individuelle summer øges. Dvs.
\begin{equation}
|\ell_n-S_n| \rightarrow 0 \quad \mbox{for} \quad n \rightarrow \infty
\end{equation}
Hvilket ifølge (\ref{polygoestowardscurve}) medfører at 
\begin{equation}
S_n \rightarrow \mathscr{L}(\gamma(t)) \quad \mbox{for} \quad n \rightarrow \infty
\end{equation}