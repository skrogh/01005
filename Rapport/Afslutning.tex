\section{Konklusion}
Vi har ved anvendelse af grundlæggende variationsteori udledt den generelle Euler-Lagrange ligning:
$$\frac{\partial L}{\partial y}-\frac{d}{dt}\frac{\partial L}{\partial \dot{y}}=0$$
Denne ligning har vi benyttet til, analytisk, at finde geodæter mellem to definerede punkter på følgene flader i rummet:
\begin{itemize}
\item Cylindere
\item Parabloider
\item Kugler
\item Æggebakker
\end{itemize}
Det blev i visse tilfælde umuligt at finde en analytisk løsning og vi har derfor anvendt værktøjet Maple til at finde numeriske løsninger til Euler-lagrange ligningen. Ydermere har vi undersøgt polygonkurver der approksimerer kontinuerte kurver, og demonstreret forskellige metoder til at udregne deres længde. Til sidst er værktøjet Maple blevet benyttet til at approksimere en geodæt på den ellers vanskelige flade, døbt "chokoladeæsken".